%==============================================================================
% Packages
%==============================================================================
\usepackage{graphicx} % Allows including images
\usepackage{booktabs} % Allows the use of \toprule, \midrule and \bottomrule in tables
\usepackage{bbm}
\usepackage{textpos}
\usepackage{tikz}
%==============================================================================
% Theorem formatting commands.
%==============================================================================
\theoremstyle{plain}
\newtheorem{thm}{Theorem}
\newtheorem{axiom}{Axiom}
\newtheorem{prop}{Proposition}
\newtheorem{cor}{Corollary}
\newtheorem{conj}{Conjecture}
\newtheorem{rul}{Rule}
\newtheorem*{thm*}{Theorem}
\newtheorem*{lemma*}{Lemma}
\newtheorem*{prop*}{Proposition}
\newtheorem*{cor*}{Corollary}
\newtheorem*{conj*}{Conjecture}

\theoremstyle{definition}
\newtheorem{defn}{Definition}
\newtheorem{ex}{Example}
\newtheorem{pr}{Problem}[section]
\newtheorem{soln}{Solution}[section]
\newtheorem{alg}[thm]{Algorithm}
\newtheorem{ques}[thm]{Question}

\theoremstyle{remark}
\newtheorem*{rmk}{Remark}
\newtheorem*{obs}{Observation}

%==============================================================================
% Blackboard Letters
%==============================================================================
%\newcommand{\aa}{\mathbb{A}}
\newcommand{\bb}{\mathbb{B}}
\newcommand{\cc}{\mathbb{C}}
\newcommand{\dd}{\mathbb{D}}
\newcommand{\ee}{\mathbb{E}}
\newcommand{\ff}{\mathbb{F}}
%\newcommand{\gg}{\mathbb{G}}
\newcommand{\hh}{\mathbb{H}}
\newcommand{\ii}{\mathbb{I}}
\newcommand{\jj}{\mathbb{J}}
\newcommand{\kk}{\mathbb{K}}
%\newcommand{\ll}{\mathbb{L}}
\newcommand{\mm}{\mathbb{M}}
\newcommand{\nn}{\mathbb{N}}
\newcommand{\oo}{\mathbb{O}}
\newcommand{\pp}{\mathbb{P}}
\newcommand{\qq}{\mathbb{Q}}
\newcommand{\rr}{\mathbb{R}}
%\newcommand{\ss}{\mathbb{S}}
%\newcommand{\tt}{\mathbb{T}}
\newcommand{\uu}{\mathbb{U}}
\newcommand{\vv}{\mathbb{V}}
\newcommand{\ww}{\mathbb{W}}
\newcommand{\xx}{\mathbb{X}}
\newcommand{\yy}{\mathbb{Y}}
\newcommand{\zz}{\mathbb{Z}}

%==============================================================================
%Caligraphics
%==============================================================================
\newcommand{\cA}{\mathcal{A}}
\newcommand{\cB}{\mathcal{B}}
\newcommand{\cC}{\mathcal{C}}
\newcommand{\cD}{\mathcal{D}}
\newcommand{\cE}{\mathcal{E}}
\newcommand{\cF}{\mathcal{F}}
\newcommand{\cG}{\mathcal{G}}
\newcommand{\cH}{\mathcal{H}}
\newcommand{\cI}{\mathcal{I}}
\newcommand{\cJ}{\mathcal{J}}
\newcommand{\cK}{\mathcal{K}}
\newcommand{\cL}{\mathcal{L}}
\newcommand{\cM}{\mathcal{M}}
\newcommand{\cN}{\mathcal{N}}
\newcommand{\cO}{\mathcal{O}}
\newcommand{\cP}{\mathcal{P}}
\newcommand{\cQ}{\mathcal{Q}}
\newcommand{\cR}{\mathcal{R}}
\newcommand{\cS}{\mathcal{S}}
\newcommand{\cT}{\mathcal{T}}
\newcommand{\cU}{\mathcal{U}}
\newcommand{\cV}{\mathcal{V}}
\newcommand{\cW}{\mathcal{W}}
\newcommand{\cX}{\mathcal{X}}
\newcommand{\cY}{\mathcal{Y}}
\newcommand{\cZ}{\mathcal{Z}}
\newcommand{\ca}{\mathcal{a}}

%==============================================================================
% Custom Commands
%==============================================================================
% \tbar{s}  -   Put a bar over a text character s.
\makeatletter
\newcommand*{\tbar}[1]{$\bar{\hbox{#1}}\m@th$}
\makeatother

% $\restr{f}{A}$  -   Restriction of function f to set A.
\newcommand\restr[2]{{% we make the whole thing an ordinary symbol
  \left.\kern-\nulldelimiterspace % automatically resize the bar with \right
  #1 % the function
  \vphantom{\big|} % pretend it's a little taller at normal size
  \right|_{#2} % this is the delimiter
  }}
		
%==============================================================================
% Custom Operators
%==============================================================================	
\DeclareMathOperator{\id}{id}
\newcommand{\Mod}[1]{\ (\mathrm{mod}\ #1)}
\DeclareMathOperator{\Span}{span}
\DeclareMathOperator{\card}{card}
\DeclareMathOperator{\Int}{Int}
